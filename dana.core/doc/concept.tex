
\begin{itemize}
\item [{\bf Excerpt from Dave Abrahams talk summary at BoostCon 07}\\
\\
]
{\em {\small "Python and C++ are in many ways as different as two languages could be:
     while C++ is usually compiled to machine-code, Python is interpreted.
     Python's dynamic type system is often cited as the foundation of its
     flexibility, while in C++ static typing is the cornerstone of its
     efficiency. C++ has an intricate and difficult compile-time meta-language,
     while in Python, practically everything happens at runtime.\\
     \\
     
     Yet for many programmers, these very differences mean that Python and C++
     complement one another perfectly. Performance bottlenecks in Python
     programs can be rewritten in C++ for maximal speed, and authors of
     powerful C++ libraries choose Python as a middleware language for its
     flexible system integration capabilities."}}
\end{itemize}

DANA is based on both C++ and Python. C++ ensures some decent speed for
simulation while python provides a powerful scripting language that can be used
for model design, interaction and visualization. The challenge is to export C++
objects to python with minimal effort and this is precisely what the boost
python library has been designed for. In the end, the user is able to build
various models simply by importing the relevant libraries into python.\\
\\

Hence, using DANA means writing some python scripts that import the core (and
possibly some other packages) in order to build and run a model. However, if
the core of DANA provides the user with a distributed numerical and
asynchronous computational paradigm, it does not provide any model at all. For
example, the core unit does not compute anything and this the responsability of
the user to write a unit class derived from the core unit that does compute
something useful.

\section{Overview}

\section{Implementation}

